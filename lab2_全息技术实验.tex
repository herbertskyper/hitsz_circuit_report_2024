\documentclass[signature=preparation]{physicsreport}

%%
%% User settings
%%

\classno{}
\stuno{}
\groupno{}
\stuname{}
\expdate{\expdatefmt\today}
\expname{全息技术实验}

%%
%% Document body
%%

\begin{document}
% First page
% Some titles and personal information are defined in ``\maketitle''.
\maketitle

\section{预习}
简述全息照相的记录与再现原理。

\newpage

\section{原始数据记录}


\begin{table*}[ht]
    \renewcommand{\arraystretch}{1.4}
    \small\selectfont
    \centering
    \caption{光路信息}
    \begin{tabularx}{\textwidth}{|c|Y|Y|Y|Y|}\hline
        物光光强 & 参考光光强  & 物光光程(dm) & 参考光光程(dm)  & 参考光与物光的夹角(°) \\\hline
                &            &               &                  &                       \\\hline
    \end{tabularx}
\end{table*}

% Teacher signature
\makeatletter
\physicsreport@body@signature{preparation}
\makeatother

\newpage

\section{实验现象分析及结论}
试分析哪些因素会对全息成像有影响。
\begin{itemize}
    \item 物光波和参考光波的光强:如果比例不合适,可能会导致全息图的对比度不足,影响成像质量。
    \item 物光波和参考光波的光程差:物光波和参考光波的光程差需要控制在一个合适的范围内。
    \item 物光波和参考光波的夹角:物光波和参考光波的夹角会影响全息图的空间频率,从而影响全息图的解析度和视场大小。
    \item 曝光时间:曝光时间过长或过短都可能导致全息图的过曝或欠曝。
\end{itemize}
\vspace*{8cm}

\section{讨论题}
\begin{enumerate}
    \item 试比较全息照相与普通照相的异同点。

    异:\begin{itemize}
        \item 全息照相不仅记录了物体的振幅信息,还记录了物体的相位信息,而普通照相只能记录物体的振幅信息。
        \item 全息照相可以重建出物体的三维信息,包括深度,而普通照相只能产生二维的图像。
        \item 全息照相需要使用激光作为光源,因为需要光源具有良好的相干性。而普通照相可以使用任何类型的光源,包括自然光。
        \item 全息照相的成像过程需要两步,先记录全息图,然后再通过适当的光源进行再现。而普通照相的图像可以直接在照相机的取景器或者显示屏上看到。
    \end{itemize}

    同:\begin{itemize}
        \item 两者都是通过捕捉光线来记录图像的技术。
        \item 两者都需要一个物理介质来记录光线信息。
    \end{itemize}

    \item 为什么用白光照射全息照片会出现彩带?为什么说观察到彩带即说明拍摄成功?
    
    \hspace*{2em}
    全息照片记录了物体的振幅和相位信息。当白光照射全息照片时,不同波长的光会因为相位差异在不同的位置上产生强度最大的亮点,形成彩带。

    \hspace*{2em}
    彩带的出现说明全息照片成功地记录了物体的相位信息。如果全息照片只记录了振幅信息,而没有记录相位信息,那么在白光照射下是不会出现彩带的。因此,观察到彩带可以作为全息拍摄成功的一个直观的标志。

    \item 参考光与物光之间夹角的大小对成像有何影响?
    
    \hspace*{2em}
    夹角会影响干涉条纹的疏密。夹角越大,全息图的空间频率越高,解析度越好,但视场越小。因此,常选用30°到40°之间的夹角。

\end{enumerate}

\end{document}