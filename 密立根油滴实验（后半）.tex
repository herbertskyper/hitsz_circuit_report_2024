\documentclass[signature=data]{physicsreport}

%%
%% User settings
%%

\classno{}
\stuno{}
\groupno{}
\stuname{}
\expdate{\expdatefmt\today}
\expname{电子电荷的测定——密立根油滴法}

%%
%% Document body
%%

\begin{document}
% First page
% Some titles and personal information are defined in ``\maketitle''.
\maketitle

\section{实验预习}


% Teacher signature
\makeatletter
\physicsreport@body@signature{preparation}
\makeatother


% Original experiment data
\section{实验现象及原始数据记录}


% Teacher signature
\makeatletter
\physicsreport@body@signature{data}
\makeatother

\newpage
% Data process and others
\section{数据处理}

    (静态法、动态法分别至少测量$3$颗油滴,计算每颗油滴的电荷量$q_i$,计算$\frac{q_i}{e}$,对商四舍五入取整后得到每颗油滴所带电子个数$n_i$;再得到每次测量的基本电荷$e_i$,再求出$n$次测量的,与理论值比较求百分误差。要有详细的计算过程,格式工整)
    
    实验中所用的有关参考数据:
    
油滴密度:$\rho=981 kg·m^{-3}$

重力加速度:$g=9.78 m·s^{-2}$

空气粘度系数:$\eta=1.83×10^{-5}·kg·m^{-1}·s^{-1}$

油滴匀速下降距离:$l=1.60×10^{-3}m$

修正常数:$b=8.22×10^{-3}m·Pa$

大气压强(深圳):$P=1.0098×10^{5} Pa$

平行极板距离:	$d=5.00×10^{-3}m$

\vspace{2em}

\section{实验结论及现象分析}
    (分析讨论本实验中出现的实验现象和电子电荷测量误差产生的原因,如何减少该误差?)
    \vspace{2em}
\section{讨论题}
\begin{enumerate}
    \item 当跟踪观察某一油滴时,原来清晰的像变模糊了,可能是什么原因造成的?
    \vspace{3cm}
    \item 由于油的挥发,油滴的质量会不断下降。当长时间跟踪测量同一个油滴时,由于油滴的挥发,会使哪些测量量发生变化。
\end{enumerate}

\end{document}