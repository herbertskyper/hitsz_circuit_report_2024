\documentclass[signature=data]{physicsreport}

%%
%% User settings
%%

\classno{}
\stuno{}
\groupno{}
\stuname{}
\expdate{\expdatefmt\today}
\expname{示波器实验}

%%
%% Document body
%%

\begin{document}
% First page
% Some titles and personal information are defined in ``\maketitle''.
\maketitle

\section{实验预习指导}
\newpage

\section{原始数据记录}
% Teacher signature
\makeatletter
\physicsreport@body@signature{data}
\makeatother

\newpage

% Data process and others
\section{数据处理}
\subsection{研究振荡频率对电力传输效率的影响}
绘制幅度-频率曲线,总结曲线规律。
\vspace{11em}

\subsection{研究无线电力传输的距离对传输效果影响}
绘制灯泡电压-距离曲线,总结曲线规律。
\vspace{11em}

\subsection{自制无线电力传输系统}
总结实际传输效果,分析误差产生的原因。

\newpage

\section{讨论题}
\subsection{为什么当振荡频率和LC电路的频率一样时,发射线圈能在周围产生大的交变磁场?}
\vspace{10em}
\subsection{你认为提高磁耦合谐振式无线电力传输系统能量传输效率的方式有哪些?}

\end{document}